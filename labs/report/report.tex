\documentclass[bachelor, och, report]{shiza}
% параметр - тип обучения - одно из значений:
%    spec     - специальность
%    bachelor - бакалавриат (по умолчанию)
%    master   - магистратура
% параметр - форма обучения - одно из значений:
%    och   - очное (по умолчанию)
%    zaoch - заочное
% параметр - тип работы - одно из значений:
%    referat    - реферат
%    coursework - курсовая работа (по умолчанию)
%    diploma    - дипломная работа
%    pract      - отчет по практике
% параметр - включение шрифта
%    times    - включение шрифта Times New Roman (если установлен)
%               по умолчанию выключен
\usepackage{subfigure}
\usepackage{tikz,pgfplots}
\pgfplotsset{compat=1.5}
\usepackage{float}

%\usepackage{titlesec}
\setcounter{secnumdepth}{4}
%\titleformat{\paragraph}
%{\normalfont\normalsize}{\theparagraph}{1em}{}
%\titlespacing*{\paragraph}
%{35.5pt}{3.25ex plus 1ex minus .2ex}{1.5ex plus .2ex}

\titleformat{\paragraph}[block]
{\hspace{1.25cm}\normalfont}
{\theparagraph}{1ex}{}
\titlespacing{\paragraph}
{0cm}{2ex plus 1ex minus .2ex}{.4ex plus.2ex}

% --------------------------------------------------------------------------%


\usepackage[T2A]{fontenc}
\usepackage[utf8]{inputenc}
\usepackage{graphicx}
\graphicspath{ {./images/} }
\usepackage{tempora}

\usepackage[sort,compress]{cite}
\usepackage{amsmath}
\usepackage{amssymb}
\usepackage{amsthm}
\usepackage{fancyvrb}
\usepackage{listings}
\usepackage{listingsutf8}
\usepackage{longtable}
\usepackage{array}
\usepackage[english,russian]{babel}

\usepackage[colorlinks=false]{hyperref}
\usepackage{url}

\usepackage{underscore}
\usepackage{setspace}
\usepackage{indentfirst} 
\usepackage{mathtools}
\usepackage{amsfonts}
\usepackage{enumitem}
\usepackage{tikz}
\usepackage{minted}

\newcommand{\eqdef}{\stackrel {\rm def}{=}}
\newcommand{\specialcell}[2][c]{%
\begin{tabular}[#1]{@{}c@{}}#2\end{tabular}}

\renewcommand\theFancyVerbLine{\small\arabic{FancyVerbLine}}

\newtheorem{lem}{Лемма}

\begin{document}

% Кафедра (в родительном падеже)
\chair{теоретических основ компьютерной безопасности и криптографии}

% Тема работы
\title{Практические задания по курсу <<Нейронные сети>>}

% Курс
\course{5}

% Группа
\group{531}

% Факультет (в родительном падеже) (по умолчанию "факультета КНиИТ")
\department{факультета КНиИТ}

% Специальность/направление код - наименование
%\napravlenie{09.03.04 "--- Программная инженерия}
%\napravlenie{010500 "--- Математическое обеспечение и администрирование информационных систем}
%\napravlenie{230100 "--- Информатика и вычислительная техника}
%\napravlenie{231000 "--- Программная инженерия}
\napravlenie{100501 "--- Компьютерная безопасность}

% Для студентки. Для работы студента следующая команда не нужна.
% \studenttitle{Студентки}

% Фамилия, имя, отчество в родительном падеже
\author{Окуньков Сергей Викторович}

% Заведующий кафедрой
% \chtitle{} % степень, звание
% \chname{}

%Научный руководитель (для реферата преподаватель проверяющий работу)
\satitle{доцент} %должность, степень, звание
\saname{И. И. Слеповичев}

% Руководитель практики от организации (только для практики,
% для остальных типов работ не используется)
% \patitle{к.ф.-м.н.}
% \paname{С.~В.~Миронов}

% Семестр (только для практики, для остальных
% типов работ не используется)
%\term{8}

% Наименование практики (только для практики, для остальных
% типов работ не используется)
%\practtype{преддипломная}

% Продолжительность практики (количество недель) (только для практики,
% для остальных типов работ не используется)
%\duration{4}

% Даты начала и окончания практики (только для практики, для остальных
% типов работ не используется)
%\practStart{30.04.2019}
%\practFinish{27.05.2019}

% Год выполнения отчета
\date{2023}

\maketitle

% Включение нумерации рисунков, формул и таблиц по разделам
% (по умолчанию - нумерация сквозная)
% (допускается оба вида нумерации)
% \secNumbering

%-------------------------------------------------------------------------------------------

\tableofcontents

\section{Задание 1: Создание ориентированного графа}
    \subsection{Описание}

        \textbf{На входе:} текстовый файл с описанием графа в виде списка дуг:
        \[(a_1, b_1, n_1), (a_2, b_2, n_2), \dots, (a_k, b_k, n_k),\] где $a_i$
        "--- начальная вершина дуги $i$, $b_i$ "--- конечная вершина дуги $i$,
        $n_i$ "--- порядковый номер дуги в списке всех заходящих в вершину $b_i$
        дуг.

        \textbf{На выходе:} Ориентированный граф с именованными вершинами и
        линейно упорядоченными дугами (в соответствии с порядком из текстового
        файла). Сообщение об ошибке в формате файла, если ошибка присутствует.

    \subsection{Пример исполнения программы}

        Содержимое файла `example.txt`:

        \begin{minted}[breaklines,fontsize=\small]{text}
            (v_1, v_2, 1), (v_2, v_3, 2), (v_1, v_4, 1)
        \end{minted}

        Запускаем программу с помощью консоли следующим образом:

        \begin{minted}[breaklines,fontsize=\small]{text}
            python3 task1.py --input_file tests/example.txt --output_file tests/example.xml
        \end{minted}

        В качестве результата получаем файл `example.xml` со следующим содержимым:

        \begin{minted}[breaklines,fontsize=\small]{text}
            <graph>
                <vertex>v_4</vertex>
                <vertex>v_3</vertex>
                <vertex>v_2</vertex>
                <vertex>v_1</vertex>
                <arc>
                    <from>v_1</from>
                    <to>v_2</to>
                    <order>1</order>
                </arc>
                <arc>
                    <from>v_1</from>
                    <to>v_4</to>
                    <order>1</order>
                </arc>
                <arc>
                    <from>v_2</from>
                    <to>v_3</to>
                    <order>2</order>
                </arc>
                </graph>         
        \end{minted}

\section{Задание 2: Создание функции по графу}
    \subsection{Описание}

        \textbf{На входе:} ориентированный граф с именованными вершинами как
        описано в задании 1.

        \textbf{На выходе:} линейное представление функции, реализуемой графом в
        префиксной скобочной записи: $$A_1(B_1(C_1(\dots), \dots, C_m(\dots)),
        \dots, B_n(\dots))$$

    \subsection{Пример исполнения программы}

    Содержимое файла `example.xml`:

        \begin{minted}[breaklines,fontsize=\small]{text}
            <graph>
                <vertex>v_4</vertex>
                <vertex>v_3</vertex>
                <vertex>v_2</vertex>
                <vertex>v_1</vertex>
                <arc>
                    <from>v_1</from>
                    <to>v_2</to>
                    <order>1</order>
                </arc>
                <arc>
                    <from>v_1</from>
                    <to>v_4</to>
                    <order>1</order>
                </arc>
                <arc>
                    <from>v_2</from>
                    <to>v_3</to>
                    <order>2</order>
                </arc>
            </graph>      
        \end{minted}

        Запускаем программу с помощью консоли следующим образом:

        \begin{minted}[breaklines,fontsize=\small]{text}
            python3 task2.py --input_file tests/example.xml --output_file tests/example2.txt
        \end{minted}

        В качестве результата получаем файл `example2.txt`' со следующим содержимым:

        \begin{minted}[breaklines,fontsize=\small]{text}
            v_1(v_2(v_3), v_4)
        \end{minted}

\section{Задание 3: Вычисление значение функции на графе}
    \subsection{Описание}

        \textbf{На входе:}
        \begin{enumerate}
            \item Текстовый файл с описанием графа в виде списка дуг (смотри задание 1).
            \item Текстовый файл соответствий арифметических операций именам вершин:
            
                \begin{center}
                    $a_1 : 1\text{-я операция}$ \\
                    $a_2 : 2\text{-я операция}$ \\
                    $\dots$ \\
                    $a_n : n\text{-я операция}$, \\
                \end{center}
                где $a_i$ -- имя $i$-й вершины, $i$-я операция -- символ операции, соответствующий вершине $a_i$.
                
                Допустимы следующие символы операций: \\
                $+$ -- cумма значений,\\
                $*$ -- произведение значений,\\
                $exp$ -- экспонирование входного значения,\\
                число -- любая числовая константа.\\		
        \end{enumerate}

        \textbf{На выходе:} значение функции, построенной по графу и файлу.
        

    \subsection{Пример исполнения программы}

        Содержимое `example.txt` и `example.json`:

        Содержимое файла `example.xml`:

        \begin{minted}[breaklines,fontsize=\small]{text}
            <graph>
                <vertex>v_4</vertex>
                <vertex>v_3</vertex>
                <vertex>v_2</vertex>
                <vertex>v_1</vertex>
                <arc>
                    <from>v_1</from>
                    <to>v_2</to>
                    <order>1</order>
                </arc>
                <arc>
                    <from>v_1</from>
                    <to>v_4</to>
                    <order>1</order>
                </arc>
                <arc>
                    <from>v_2</from>
                    <to>v_3</to>
                    <order>2</order>
                </arc>
            </graph>      
        \end{minted}


        \textbf{example.json}
        \begin{minted}[breaklines,fontsize=\small]{text}
            {
                "v_1": "+",
                "v_2": "exp",
                "v_3": 5,
                "v_4": 8
            }
        \end{minted}
        
        Запускаем программу с помощью консоли следующим образом:

        \begin{minted}[breaklines,fontsize=\small]{text}
            python3 task3.py --input_file tests/example.xml --output_file tests/example3.txt --nodes_to_func_file tests/example.json
        \end{minted}

        В качестве результата получаем файл `example3.txt` со следующим содержимым:

        \begin{minted}[breaklines,fontsize=\small]{text}
            [156.4131591025766]
        \end{minted}

\section{Задание 4: Построение многослойной нейронной сети}
    \subsection{Описание}

        \textbf{На входе:}
            \begin{enumerate}
                \item Файл с набором матриц весов межнейронных связей:
                \begin{center}
                    $M_1 : [a_{11}^1, a_{12}^1, \dots, a_{1n_1}^1], \dots, [a_{m_11}^1, a_{m_12}^1, \dots ,a_{m_1n_1}^1]$ \\
                    $M_2 : [a_{11}^2, a_{12}^2, \dots, a_{1n_2}^2], \dots, [a_{m_21}^2, a_{m_22}^2, \dots ,a_{m_2n_2}^2]$\\
                    $\dots$\\
                    $M_p : [a_{11}^p, a_{12}^p, \dots, a_{1n_p}^p], \dots, [a_{m_p1}^p, a_{m_p2}^p, \dots,a_{m_pn_p}^p]$ \\                  
                \end{center}
                \item Файл с входным вектором в формате:
                \begin{center}
                    $x_1, x_2, \dots, x_k$.
                \end{center}
            \end{enumerate}

        \textbf{На выходе:}
            \begin{enumerate}
                \item Сериализованная многослойная нейронная сеть с полносвязной
                межслойной структурой. Файл с выходным вектором -- результатом
                вычислений НС в формате: 
                \begin{center}
                    $y_1, y_2, \dots, y_k.$                
                \end{center}
                \item Сообщение об ошибке, если в формате входного вектора или
                файла описания НС допущена ошибка.
            \end{enumerate}


    \subsection{Пример исполнения программы}

        Содержимое файлов `nn.json` и `X.txt`:

        \textbf{X.txt}
        \begin{minted}[breaklines,fontsize=\small]{text}
            1., 2., 3.
        \end{minted}
        
        \textbf{nn.json}
        \begin{minted}[breaklines,fontsize=\small]{text}
            {
                "layer1.sigmoid": [[1, 1, 1], [2, 2, 2]],
                "layer2.linear": [[1, 1], [2, 2], [3, 3]],
                "layer3.softmax": null,
                "layer4.argmax": null
            }
        \end{minted}

        Запускаем программу с помощью консоли следующим образом:

        \begin{minted}[breaklines,fontsize=\small]{text}
            python3 task4.py --input_file tests/X.txt --weights_file tests/nn.json --output_file tests/Y.txt
        \end{minted}

        В качестве результата получаем файл `Y.txt` со следующим содержимым:

        \begin{minted}[breaklines,fontsize=\small]{text}
            {"Y": 2}
        \end{minted}

\end{document}